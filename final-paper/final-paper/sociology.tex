\section{Previous Literature and Sociological Reasons for Broadband's Effect on Economic Growth}
\label{sec:sociology}

There are two major trends in the literature on the economic effects of broadband.
First, many papers from both academia and industrial research consider
economic benefits for individual firms and communities
within OECD (developed) countries. For example, a study by MIT measured
qualitative and quantitative differences over time between communities in the
United States who adopted early broadband technologies, versus communities who
were late to adopt the technology~\cite{gillett2006measuring}.
At least one study, similar in spirit but not in scope to the World Bank
study, looked at the macroeconomic effects of broadband penetration across
20 OECD countries~\cite{koutroumpis2009economic}. Nonetheless, relatively few
studies have examined the effects of broadband on developing
economies~\cite{qiang2010broadband}, the main focus of this paper.

A second line of related work examined the extent to which broadband technologies
are economically preferable to narrowband technologies such as cell phones or
dial-up. For example, one study found that the typical broadband user spends 64\%
more time on Internet the typical narrowband user~\cite{saksena2003igniting}.
I touch on the macroeconomic differences between broadband and narrowband in
\S\ref{sec:results}, but it is not the main purpose of this paper.

Regardless of positioning ourselves against related work, it is helpful to
consider the microeconomic and sociological reasons that underly the
belief that broadband spurs macroeconomic growth. Following Quiang et.
al~\cite{qiang2010broadband}, I outline four scopes of sociological effects
induced by broadband: \\

\noindent{}{\bf Benefits for individuals}. As an information technology, broadband
 has the potential to increase the human capital stock of the countries that deploy it;
 with high-speed Internet connections, individuals have the opportunity to enhance their
 knowledge and skills through resources such as online courses, blogs and wikis~\cite{johnson2005next}.
 Even access to recreational technologies such social networking sites has the potential to grow individuals'
 social networks and expose them to different cultural values. Finally, broadband access has the potential to
 empower individuals to drive innovations of products based on their own needs, rather than waiting
 for institutional forces to make the changes for them.  For example, high-speed Internet
 enables participation in community-lead open source projects~\cite{von2009democratizing}.

\noindent{}{\bf Benefits for firms}. Broadband Internet access has immediate potential to lower costs
 and raise productivity for private companies. For example, one paper cites that a collection of U.S. firms
 were able to save \$155 billion in operating costs as a result deploying broadband~\cite{varian2002net}.
 The reasons for reduced costs depend on the company's business model. One company in Britain, for example,
 was able to save 60 million pounds in employee's medical expenses and overall productivity by using broadband
 to allow employees to work from home and connect remotely to their internal computing resources~\cite{bband_stakeholder}.

 In more extreme cases, broadband gives companies the potential to completely alter their business models.
 For example, industries with products that can be distributed entirely through electronic means, such as the
 advertising, movie, music, or gaming industries, can (and do) completely change their business strategies when their
 customers have access to high-speed Internet~\cite{heng2006media}. Moreover, broadband yields access to
 foreign markets that companies would otherwise not have been able to reach; one study found that if the number of
 companies with Internet access increases by 1 percent, there is a corresponding boost of 4.3 percent in exports, and a
 3.8 percent boost in exports from companies in low-income counties selling to high-income countries~\cite{clarke2004has}.

\noindent{}{\bf Benefits for communities}. Broadband also plays a role in
creating stronger communities at the municipality level. For example, if a
community subsidizes residential and commercial broadband deployment, it
stands to foster a better environment for business to grow
in. A study from MIT compared early-adopters
of broadband versus late-adopters between 1998 and 2001, and found that
communities with broadband experienced higher employment growth rates, more
new businesses, and higher
rent prices (a proxy for the perceived value of living in the community), than
communities that did not adopt broadband~\cite{gillett2006measuring}.
Broadband also plays a crucial role in reducing information assymetry (where
customers do not have sufficient information to make informed buying
decisions), thereby creating a more effecient market for both buyers and
sellers. Lastly, broadband helps communities deploy public services that
otherwise might be intractable. For example, services such as telemedicine and online education
benefit the overall community at a low cost if broadband is pervasive. This is
especially relevent for remote communities; consider, for example, the use of telemedicine
 to reach villages in the Upper Amazon~\cite{wootton2010circumstances}, where it would otherwise
 take three days by canoe to bring a sick or injured person to the nearest doctor.

\noindent{}{\bf Benefits for the overall economy}. Broadband deployment in theory may
carry positive externalities with it that affect the overall economy.
Several speculative studies have predicted the macroeconomic effects of broadband
on individual countries. One study by a technology company research lab\footnote{An admittedly vested interest}
 found in 2003 that increased broadband deployment had the potential to contributed \$500 Billion to the United States
 economy~\cite{saksena2003igniting}.

 Besides the microeconomic reasons already described, there are several reasons why broadband deployment
 might yield macroeconomic benefits. One major reason is that access to the Internet facilitates overall trade with foreign economies
 in services and globalization~\cite{attendusing}. In support of this reason, it has been shown that the existence of broadband
 is an important factor in whether foreign countries decide to place investments in growing markets~\cite{abramovsky2006outsourcing}.
 Broadband also stands to transforming research and development, the means by which countries make technological progress (a factor
 which is crucial in exogenous model of economic growth). One study in this space surveyed the role broadband plays in generating
 inventions~\cite{carlaw2007past}. Another cited that broadband encourages collaboration, access to data, and round-the-clock
 development~\cite{van2008broadband}. Lastly, broadband can help governements streamline public services, thereby facilitating
 a more conducive business environment. E-governance, and electronic customs processes, are examples of such streamlined
 services~\cite{de2004customs}

Lastly, broadband technologies arguably have benefits that extend past direct economic effects. For example,
some have argued that Internet access as played a role in organizing political revolutions and spreading
democracy~\cite{zhang2010revolution}. This line of inquiry is outside the scope of this paper.
