\section{Previous Literature and Sociological Reasons for Broadband's Effect on Economic Growth}
\label{sec:sociology}

% REWORD:
Previous literature has examined the macroeconomic effect of GDP in OECD
(developed) countries~\cite{koutroumpis2009economic}, as well as individual
communities~\cite{gillett2006measuring} and
qualitative benefits for individual firms~\cite{varian2002net} in OECD
countries.

- Lots of related work on economic benefits in *developed* countries. MIT comparative study on communities who were earlty adopters of broadband vs. non-adopters ~\cite{gillett2006measuring}, and macro-economic analysis of 20 OECD countries (similar to Quiang’s analysis) ~\cite{koutroumpis2009economic} -- very little on developing regions ~\cite{Quiang}.
- Work on whether broadband is in fact economically preferable to narrowband
    - Spend 64\% more time on Internet ~\cite{saksena2003igniting}


Quiang et. al outlined four scropes of sociological : 
   - Human capital: enhancing skills, knowledge, and social network of individuals
       - blogs,wikis,etc. engender knowledge ~\cite{johnson2005next}
       - facilitates individual or user-led (rather than institutional) innovation, e.g. open source projects ~\cite{von2009democratizing}

   - raising private sector efficiency
       - lower costs and raise productivity (numbers in paper) ~\cite{varian2002net}
       - working from home saved 60M pounds ~\cite{bband_stakeholder}
       - Completely change the business model, e.g. revolutionizing movie, music, gaming, advertising industries with customers ~\cite{heng2006media}
       - increased access to foreign markets: ``1 percentage point increase in
       the number of Internet users is correlated with a boost in exports of
       4.3 percentage points and an increases in exports from low-income to
       high-income countries of 3.8 percentage points''~\cite{clarke2004has}
   - increasing community competitiveness
        - reducing information asymmetry, a source of economic inefficiency (no cite)
        - creates a better environment for business to grow in: ``between 1998
        and 2001, early adopters of mass-market broadband experienced faster
        growth in employment, number of businesses, and businesses in
        IT-intensive sectors, as well as higher market rates for rental
        housing, than communi- ties where broadband was adopted later'' (numbers in paper)~\cite{gillett2006measuring}
          - enabling new industires not previously possible, e.g. telemedicine and online education
 
- And a few positive externalities:
   - A few studies of per-country benefits of GDP 
   - transforming Research and Development. Crucial for generating inventions ~\cite{carlaw2007past}. 
``It allows around-the-clock Research and Development and concurrent
development on multiple phases and projects in different locations.'' and encourages collaboration, access to data ~\cite{van2008broadband}
   - facilitating trade in services and globalization -- existence of broadband an important factor in deciding whether to invest foreign money in a counry ~\cite{abramovsky2006outsourcing}. broadband encourages trade ~\cite{attendusing}
   - improve public services (thereby facilitating a conducive business environment): e-governance, customs ~\cite{de2004customs}

And some non-economic categories?:
   - political activism
   - spread of democracy

