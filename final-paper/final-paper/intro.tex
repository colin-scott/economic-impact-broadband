In this paper I attempt to answer a relatively straightforward question: does
broadband Internet access spur economic growth? Here, broadband refers to any
Internet connection that is ``always on and faster than the traditional dial-up
access.''~\cite{fcc_bband}, and economic growth refers to gross domestic
product (GDP) of individual countries\footnote{This paper does not consider
a microeconomic analysis of broadband access}.

Intuitively, it seems clear that access to broadband Internet would engender
economic growth. For example, access to
broadband allows individuals educate themselves (thereby contributing to a
nations overall human capital stock), helps organizations streamline business
procedures and cut costs, yields access to new buyers and sellers in remote markets,
enables new business models (consider, for example, telemedicine),
facilitates more efficient market prices by reducing
information asymmetry, forms an important part of the overall business environment
that is conducive to growth, and catalyzes the forming of
organizations between individuals who normally wouldn’t be connected.

Indeed, the ICTD research community generally believes that broadband
spurs economic growth~\cite{brewer2005case}. In fact, several groups are leading major
projects based on this assumption. Berkeley's Tier group~\cite{tier} is working on
building cheaper, easier to deploy long-distance wifi technologies. University of California Santa Barbara has a
group looking at satellite and cellular technologies for rural communities~\cite{ucsb}.
IIT Delhi in India is working on content distribution networks for rural areas
in India~\cite{mahla2012motivation}, and New York University is working on
maintenence challenges in rural broadband networks~\cite{surana2008beyond}.
If their assumption is true, their results stand to directly impact global
development goals development goals; consider that over 70\% of the world, including
30\% of the U.S. population, lives without Internet, primarily in rural areas~\cite{rural_stats}.

Likewise, many policy decisions are based around assumption of broadband's
economic value. Many countries already
believe that broadband has positive benefits, and seek to subsidize
deployment. In 2008, for example, the government of Brazil worked with
five wired broadband providers to build a broadband network to connect public
schools in over 3,000 municipalities by the end of 2010 ~\cite{gazeta}.

Depsite these efforts, there is relatively little empirical data in support of this belief!
Several research studies have looked at developed countries, and individual
firms that have benefited from broadband. XXX
But few have performed a broad analysis across both developed and developing
countries. The only substantial study I know in support of this belief is a World Bank report
from 2009, which found that a 10 percent increase in broadband penetration
correlates with a 1.3 percent increase in GDP~\cite{qiang2009economic}. This
finding was based on an econometric analysis of the following
variables: per capita GDP, ratio of
investment to GDP, primary school enrollment (a proxy for human capital stock),
and average penetration of broadband services~\footnote{The data was drawn
from the International Telecommunication Union World
Telecommunication Indicators Database (2007) and World Bank World Development Indicators Online
Database (2008) over 120 developing
and developed countries}.

I hypothesize that this finding does not generalize: growth in GDP is not
strongly affected by broadband Internet penetration (alone). I propose to
support my hypothesis by performing a similar econometric analysis. I will
start by performing the same calculations on up-to-date data from the same
sources to see if
Quiang's results are corroborated. I will then attempt to examine alternate
variables and alternate data sources.

% Regardless of the results, this project
% will inform the motivation of my research interests in deploying broadband
% wireless to rural areas.

