In this paper I attempt to answer a relatively straightforward question: does
broadband Internet access spur economic growth? Here, broadband refers to any
Internet connection that is ``always on and faster than the traditional dial-up
access.''~\cite{fcc_bband}, and economic growth refers to gross domestic
product (GDP) of individual countries\footnote{This paper does not consider
a microeconomic analysis of broadband access}.

Intuitively, it seems clear that access to broadband Internet should engender
economic growth. For example, access to
broadband allows individuals educate themselves (thereby contributing to a
nations overall human capital stock), helps organizations streamline business
procedures and cut costs, yields access to new buyers and sellers in remote markets,
enables new business models (consider, for example, telemedicine),
facilitates more efficient market prices by reducing
information asymmetry, forms an important part of the overall business environment
that is conducive to growth, and catalyzes the forming of
organizations between individuals who normally wouldn’t be connected.

Indeed, the ICTD research community generally believes that broadband
spurs economic growth~\cite{brewer2005case}. In fact, several groups are leading major
projects based on this assumption. Berkeley's Tier group~\cite{tier} is working on
building cheaper, easier to deploy long-distance wifi technologies. University of California Santa Barbara has a
group looking at satellite and cellular technologies for rural communities~\cite{ucsb}.
IIT Delhi in India is working on content distribution networks for rural areas
in India~\cite{mahla2012motivation}, and New York University is working on
maintanence challenges in rural broadband networks~\cite{surana2008beyond}.
If their assumption is true, their results stand to directly impact global
development goals development goals; consider that over 70\% of the world, including
30\% of the U.S. population, lives without Internet, primarily in rural areas~\cite{rural_stats}.

Likewise, many policy decisions are based around the assumption of broadband's
economic value. Several countries already have subsidized broadband deployments
in order to encourage economic growth. In 2008, for example, the government of Brazil worked with
five wired broadband providers to build a broadband network to connect public
schools in over 3,000 municipalities by the end of 2010 ~\cite{gazeta}.

Depsite these efforts, there is relatively little empirical data in support of this belief!
Previous literature has examined the macroeconomic effect of GDP in OECD
(developed) countries~\cite{koutroumpis2009economic}, as well as individual
communities~\cite{gillett2006measuring} and
qualitative benefits for individual firms~\cite{varian2002net} in OECD
countries. But few studies have performed a broad macroeconomic analysis
across both developed and developing
countries. The only substantial study I know of is a World Bank report
from 2009, which found that a 10 percent increase in broadband penetration
correlates with a 1.3 percent increase in GDP~\cite{qiang2009economic}.

One major reason for the sparsity of empirical studies in this area is that
broadband is a relatively recent technology; it first started appearing in
developed countries in the late 1990's, and has only become pervasive in
developing countries in the last few years. The World Bank study itself
used data from 2007, when broadband was just beginning to appear in
the developing world.

With 5 years of hindsight since the World Bank study, we are in a much better
position to evaluate the long-term economic effects of broadband penetration.
In this paper I seek to revisit the World Bank's macroeconomic analysis of
broadband's effect on GDP, hoping to corrobarate their results.

More specifically, in this paper I perform an econometric analysis which is
equivalent to the World Bank study, but includes 5 years of additional data
dating through 2011. The World Bank's study performed their econometric analysis
over the following variables: per capita GDP, ratio of
investment to GDP, primary school enrollment (a proxy for human capital stock),
and average penetration of broadband services. I examine the same variables (and others),
drawn from the same data sources: the International Telecommunication Union World
Telecommunication Indicators Database~\cite{itu} and the World Bank World Development Indicators Online
Database~\cite{wdi}, covering over 120 developing
and developed countries since 1980.

My main finding is that XXX.

The rest of this paper is organized as follows. In section \S\ref{sec:sociology} I review related work
 and outline sociological reasons for broadband's effect on economic growth. In section
\S\ref{sec:data_and_methodology}, I describe the dataset and methodology used. Lastly, in section
\S\ref{sec:results} I present my results, and in \S\ref{sec:conclusion} I conclude.

% Regardless of the results, this project
% will inform the motivation of my research interests in deploying broadband
% wireless to rural areas.

