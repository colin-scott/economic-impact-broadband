\documentclass[10pt]{article}

\usepackage{amsmath, amssymb, amsthm}
\usepackage{graphicx}	% use to include graphics
\usepackage{verbatim}
\usepackage{multicol}

\sloppy
\makeatletter
\renewcommand\section{\@startsection{section}{1}{\z@}%
								 {-3.5ex \@plus -1ex \@minus -.2ex}%
								 {2.3ex \@plus.2ex}%
								 {\normalfont\large\bfseries}}
\makeatother

\title{Does broadband Internet connectivity actually spur economic growth?}
\author{Colin Scott}

\begin{document}
\maketitle

Despite sparse empirical data, the ICTD community generally believes that access
to broadband Internet in developing regions brings about economic growth~\cite{brewer2005case}.
The only substantial study I know in support of this belief is a World Bank report
from 2009, which found that a 10 percent increase in broadband penetration
correlates with a 1.3 percent increase in GDP~\cite{qiang2009economic}. This
finding was based on an econometric analysis of the following
variables: per capita GDP, ratio of
investment to GDP, primary school enrollment (a proxy for human capital stock),
and average penetration of broadband services~\footnote{The data was drawn
from the International Telecommunication Union World
Telecommunication Indicators Database (2007) and World Bank World Development Indicators Online
Database (2008) over 120 developing
and developed countries}.

I hypothesize that this finding does not generalize: growth in GDP is not
strongly affected by broadband Internet penetration (alone). I propose to
support my hypothesis by performing a similar econometric analysis. I will
start by performing the same calculations on up-to-date data from the same
sources to see if
Quiang's results are corroborated. I will then attempt to examine alternate
variables and alternate data sources.

Regardless of the results, this project
will inform the motivation of my research interests in deploying broadband
wireless to rural areas.

\bibliographystyle{ieeetr}
\bibliography{proposal}

\end{document}
