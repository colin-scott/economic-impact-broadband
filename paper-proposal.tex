\documentclass[12pt]{article}

\usepackage{amsmath, amssymb, amsthm}
\usepackage{graphicx}	% use to include graphics
\usepackage{verbatim}
\usepackage{multicol}

\makeatletter
\renewcommand\section{\@startsection{section}{1}{\z@}%
								 {-3.5ex \@plus -1ex \@minus -.2ex}%
								 {2.3ex \@plus.2ex}%
								 {\normalfont\large\bfseries}}
\makeatother

\title{Economic Implications of Global Internet Governance}
\author{
Colin Scott
}

%%%%%%%%%%%%%%%%%%%%%%%%%%%%%%%%%%%%%%%%%%%%%%%%%%%%%%%%%%%%%%%%%%%%%%%%%%%%%%%%%%%%%
% ICANN: in charge of names and addresses. United States Department of
% Commerce (National Telecommunications and Information Administration)
% currently has the final say over changes to the DNS root zone.

% IETF: in charge of defining standards

% WGIG: created in reaction to ICANN being American (2003)

% "Domain tax" to support ICANN

% Examples of policy differences:
%  - developing countries want fettered content cencorship

% Could I use measurements (DNS, perhaps) to inform the cost of
% addresses/domain names?
\begin{document}
\maketitle

The Internetd you do9l
Internet governance dictates IP address useage, DNS names, etc. These are
important for X, Y. ICANN has traditionally been in charge of this, but there
is a recent push to globalize the governance of the Internet [cite].

Perhaps, look at cost of NAT / other address exhaustion for developing
countries. -> Potential benefits to be had from global governance, non-ASCII
DNS names, 

Argues that layer 7 is where the standards really effect developing countries.
Intellectual property burdens developing regions more than developed?

\bibliographystyle{ieeetr}
\bibliography{proposal}

\end{document}
